% PhyHubThesis v0.3.0
% GitHub地址: https://github.com/HoyanMok/PhyHubThesis
% 如GitHub不能访问, 请以邮件联系作者: victoriesmo@hotmail.com
\documentclass{ctexart} 
% \documentclass{article} 如果用英文
% \documentclass[10pt,a4paper]{ctexart}  字体大小和纸张大小,默认分别为10pt和letterpaper
% 五号 = 10.5pt,小四=12pt,四号=14pt
% 其他可选参量如twocolumn, 两行排版

\usepackage{PhyHubThesis}

% 文章标题页信息:
\title{标题}
\author{ 作者1\thanks{作者1的联系方式}
	\and
		作者2\thanks{作者2的联系方式}}
\date{\today} % 自动生成日期

\addbibresource{PhyHubThesis.bib} % 把这里改成实际的文件名

% \geometry{left=1.6cm,right=1.6cm}

\begin{document}
\maketitle % 打印标题

\begin{abstract}
	摘要内容摘要内容摘要内容摘要内容摘要内容摘要内容摘要内容

	\centering % 使得关键字居中
	\textbf{关键字:}
	% \texbf{Keywords: }
		关键字1\ 
		关键字2\  
		关键字3
\end{abstract}

\tableofcontents
\newpage

\section{模板使用说明}
\subsection{编译}
把导言区的\verb|\newcommand{\PATH}{路径}|改成此文档的相对路径 (MAC OS或Linux) 或绝对路径 (Windows), 然后在该目录下使用命令行依次运行:\\[1pt]
\texttt{xelatex} 文件名 (可以省略.tex)\\
\texttt{biber} 文件名 (可以省略.bib)\\
\texttt{xelatex} 文件名\\
\texttt{xelatex} 文件名\\

\subsection{使用示例}

引用.bib中的书目: \cite{Knuth}.

\begin{figure}[h] %
\centering
\begin{subfigure}{.5\textwidth}
	\centering
	\includegraphics[width=6cm]{latex-project-logo.pdf}
	\caption{题注}
	\label{子图1}
\end{subfigure}%
\begin{subfigure}{.5\textwidth}
	\centering
	\includegraphics[width=6cm]{latex-project-logo.pdf}
	\caption{题注}
	\label{子图2}
\end{subfigure}
\caption{一个并排放置图片的示例的示例}
\end{figure}

交叉引用的例子: 图~\ref{子图1} 是\LaTeX{} Project的图标. 

单行公式:
\begin{equation}\label{公式的标签}
	\int \diff[2] y x\,\dif x = \diff y x + C. 
\end{equation}

多行公式与对齐:
\begin{align}
	\nabla\cdot \bv D &= \rho
	\\
	\nabla \times \bv E & =  - \pdiff{\bv B}{t}
	\\
	\nabla \cdot \bv B &= 0
	\\
	\nabla \times \bv H &= \bv j + \pdiff{\bv D}{t}
\end{align}

\begin{figure}[H]%使用figure环境
	\begin{center}
	  \begin{circuitikz}
		\draw (0,0) % 坐标 (0, 0) 做为起始点,(0, 2) 做为终点,绘制电压源。
		% V 代表电压源,v=$U_q$ 绘制标识。
		to[V, v=$U_q$] (0,2) % 电压源
		to[short] (2,2) % 坐标(2,2)做为起始点,(2,0)做为终点,绘制电阻。_R_代表电压源,_R=$R_1$_绘制标识。
		to[R=$R_1$] (2,0) % 电阻
		to[short] (0, 0); % 注意结尾的分号!
	  \end{circuitikz}
	  \caption{first circuit.}%添加标题
	\end{center}
  \end{figure}

\section{\texorpdfstring{在目录和文档中显示的标题}{在书签中显示的标题}}

可以使用\verb|\texorpdfstring{在目录和文档中显示的标题}{在书签中显示的标题}|来处理标题上有\LaTeX{}公式时标签不能正常显示的问题.

\appendix
\section{某附录}
\nocite{*} % 这个表示列出所有没有在文中被引用的参考文献
\printbibliography[heading=withoutstar, title={参考文献}] % 如heading=withstar则会将参考文献前的 (附录编号) 字母削除
% \printbibliography[heading=withoutstar, title={Bibliography}]
\end{document}